\chapter{Dealing with files}

In standard C, a file is represented by a \texttt{FILE *} (file pointer).
The contents of this type are unimportant to the user, and you should never dereference a file pointer.

To make things convenient, the standard IO streams already have a \texttt{FILE *} globally defined in the \texttt{stdio.h} header.
Standard output is available through \texttt{stdout}, standard input is available through \texttt{stdin} and standard error is available through \texttt{stderr}.

\section{Opening a file}

The first thing you will need to do before you read/write to a file is to get a \texttt{FILE *} to it.
This is done using the \texttt{fopen} function.

\begin{codeinline}
FILE *fopen(char *name, char *mode)
\end{codeinline}

The function takes two arguments, \texttt{name} and \texttt{mode}, and returns a \texttt{FILE *}.
The argument \texttt{name} is self explanatory. It is a string containing the name of the file to be opened.
The \texttt{mode} argument describes what mode you want to open the file in.
The following \texttt{mode} values are most typical:

\begin{itemize}
  \item{\textbf{``r''}:} read mode - In this mode the file is opened for reading only, attempts to write to it will fail.
  \item{\textbf{``w''}:} write mode - In this mode the file is opened for writing only, attempts to read from it will fail (be careful, this mode removes any existing content from the file!).
  \item{\textbf{``a''}:} append mode - In this mode the file is opened for appending. This is like write mode only it does not remove the existing contents.
\end{itemize}

Opening a file can fail for many reasons, such as it not existing or not having the correct permissions.
When using \texttt{fopen}, failure is reported by returning \texttt{NULL}.
You should always check the return value before continuing.
The following example shows how to do this.

\begin{codeblock}
/* Try and open testfile.txt for reading. */ 
FILE *testFile = fopen("testfile.txt", "r");
if (testFile == NULL) {
    printf("Opening file failed!\n");
}
\end{codeblock}

\section{Reading from a file}

If you have a \texttt{FILE *} which is opened for reading, the next thing you need to do is read from it.
There are a number of functions which can be used for this, however the most useful is typically \texttt{fgetc}.

\begin{codeinline}
int fgetc(FILE *file)
\end{codeinline}

The function takes the \texttt{FILE *} you want to read from and returns the next value in the stream.
One thing that is important to note here is that while the next value in the stream is always of type \texttt{char}, \texttt{fgetc} returns type \texttt{int}. 
The reason for this is that when there is nothing left to read, the function needs a way to tell you.
It does this by returning the value \texttt{EOF}, however it needs an extra value (outside the range of a \texttt{char}) to do this.
By returning an \texttt{int} it as able to do this, however you must be careful to never directly assign the result of \texttt{fgetc} to a \texttt{char}.
The example below shows how to correctly read a character using \texttt{fgetc}.

\begin{codeblock}
int next = fgetc(stdin);
char result;
if (next == EOF) {
    printf("End of file encountered, exiting!\n");
    exit(0);
} else {
  result = (char)next;
}
\end{codeblock}

Often in your program you will want to read a line of text from the file, not just a single character.
It is helpful in this instance to write a small function that does this which you can reuse later on.
The following is a small example of how to do this.
It first checks for either end of file or newline and returns the result (making sure to terminate the string first!), otherwise it appends to new character to the string.

This example uses a fixed size buffer, it is up to the user to work out how to make this more useful.
The \texttt{realloc} function may be useful for this.
It may also be useful to differentiate between an empty line ending in a newline or a line only containing \texttt{EOF} (perhaps by returning \texttt{NULL}).

\begin{codeblock}
char *
read_line(FILE *file)
{
    char *result = malloc(sizeof(char) * 80);
    int position = 0;
    int next = 0;

    while (1) {
        next = fgetc(file);
        if (next == EOF || next == '\n') {
            result[position] = '\0';
            return result;
        } else {
            result[position++] = (char)next;
        }
    }
}
\end{codeblock}

\section{Writing to a file}

Being able to write to a file is just as important as being able to read from a file.
Fortunately this is just as easy as printing to the screen!
All you have to do is replace \texttt{printf} with \texttt{fprintf}.

\begin{codeinline}
int fprintf(FILE *file, char *fmt, ...)
\end{codeinline}

The only difference between the two is that \texttt{fprintf} allows you to specify the file that it prints to.

There is one issue that you must be aware of when writing to files, and that is \emph{buffering}. 
Writing to a file is typically expensive, so the C library tries to be smart and avoid doing it whenever possible.
It does this by buffering all the things you've written to the file until you call the \texttt{fflush} function or close the file (coming in the next section).

\begin{codeinline}
int fflush(FILE *file)
\end{codeinline}

While this buffering is good for performance, it can sometimes cause you to think your code is not printing something, when in fact it is just being buffered.
In general, whenever you want something to appear in the file (this could be stdout), you should \texttt{fflush} it after printing.

\section{When you're finished with a file}

There are only a finite amount of files a process/user can have open at any one time.
When you're finished with a file, it is good practice to close it and free up its resources.
This is done using \texttt{fclose}.
Closing the file also flushes any buffers associated with it if you hadn't done it already.

\begin{codeinline}
int fclose(FILE *file)
\end{codeinline}

After calling fclose on a \texttt{FILE *} you should no longer attempt to use it again until you reassign it using \texttt{fopen}.

\section{Complete example}

Below is a complete example you can run. 
It uses concepts from above to read a line from the user and write it to the file ``testfileio.out''.

\begin{codeblock}
#include <stdlib.h>
#include <stdio.h>
#include <string.h>

char *
read_line(FILE *file)
{
    char *result = malloc(sizeof(char) * 80);
    int position = 0;
    int next = 0;

    while (1) {
        next = fgetc(file);
        if (next == EOF || next == '\n') {
            result[position] = '\0';
            return result;
        } else {
            result[position++] = (char)next;
        }
    }
}

int
main(int argc, char **argv)
{
    FILE *outputFile;
    char *input;

    outputFile = fopen("testfileio.out", "w");

    printf("Type something: ");
    fflush(stdout); /* Appear now! */

    input = read_line(stdin);

    fprintf(outputFile, "Read \"%s\" from user!\n", input);
    fflush(outputFile);
    fclose(outputFile);

    printf("Done!\n");

    return 0;
}
\end{codeblock}
