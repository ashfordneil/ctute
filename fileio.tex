
\chapter{File IO}

Files in standard C are represented by \texttt{FILE*}.\footnote{Do not try to follow a \texttt{FILE*}, there is no point.}
The input/output functions you have been using have file versions.
Aside from taking a \texttt{FILE*} as an extra first parameter, they work in the same way.

The standard io steams already have \texttt{FILE*} variables defined for them.
\texttt{stdin} is standard input, \texttt{stdout} is standard output and \texttt{stderr} is the standard error stream. 
For example:
\lstinline!printf("%s", "Hello world")! is equivalent to \lstinline!fprintf(stdout, "%s", "Hello world")!.\\
\lstinline!scanf("%d", n[x])! is equivalent to \lstinline!fscanf(stdin, "%d", n[x])!.\\

To access external files, use \texttt{fopen} to get \texttt{FILE*} and \texttt{fclose} when you are finished.
For example:
\begin{codeinline}
FILE* f=fopen("message.txt", "r");
/* some code */
fprintf(f, "%s", "Greetings\n");
/* more code */
fclose(f);
\end{codeinline}
The second parameter on \texttt{fopen}, indicates whether the file should be opened for reading or writing\footnote{see documentation for other options}.
The code above, assumes that the open succeeded, but what if the file isn't there?
\texttt{fopen} will return a NULL pointer if the open fails, so we could test for that with something like:\\
\lstinline!if (f==NULL)! or \lstinline!if (f==0)! or \lstinline@if (!f)@

Another aspect of standard C io which is important to understand is \emph{buffering}.
Output statements\footnote{fprintf, fputc, \ldots} may execute without the output actually being sent.
Instead the output is saved up (buffered) until later when it is sent in one big chunk.
This is called \emph{flushing}.
In many cases, this is a good thing because it is more efficient.
But if you want to ensure output has actually been sent, you can force the stream to be flushed by calling \texttt{fflush}.
\begin{codeinline}
FILE* f=fopen("message.txt", "r");
if (!f) {
    /* do something if file doesn't open */
}
/* some code */
fprintf(f, "%s", "Greetings\n");
fflush(f);
/* more code */
fclose(f);
\end{codeinline}

\emph{Do not \emph{fflush} after every output statement} unless you have a good reason, it slows down your program.
Streams are \texttt{fflush}ed automatically when files are \texttt{fclose}d or when a program exits normally.

\hypertarget{disc:reversenums}{}Listing~\ref{lst:reversenums} contains an example program which reads integers from a file and then outputs them to another
file in reverse order.
To test this code (as well as entering and compiling it) you will need a text file with some integers (and whitespace) in it.
Things to note in this code:
\begin{itemize}
 \item The use of \lstinline!fprintf(stderr, ! to display error messages.
 \item Aside from the error messages, all input and ouput is done in terms of \texttt{FILE*} variables.
 This makes the code more flexible, since it would be trivial to send output to (for example) standard out, just
 by changing the argumens to \texttt{output\_nums}.
 \item there are two versions of code to read the numbers from the file.
 Switch between them by changing Line~65.
 In both cases, since the function needs to return both the array of \texttt{int}s and howmany integers were
 in the file, the count of values is sent back via a pointer.
 The \texttt{read\_nums} function reads an int at a time and stores them in a dynamic array (keeping track both of how big the array is
 and how many numbers are actually in it.
 If the array fills up, a larger array is allocated, the old values are copied over and then the old array is \texttt{free}d.
 For a slightly different approach to the resizing, see the documentation for the standard \texttt{realloc} function.
 
 The \texttt{read\_nums\_presize} function tries to work out how big the array needs to be before allocating it (avoiding the need to resize).
 It does this by reading through the whole file, counting numbers as it goes, then allocates an array, goes back to the beginning of the file
 and reads all the numbers into the array.
 This implementation is here to demonstrate the \texttt{fseek} function and because students often try to do something like this.
 It certainly works in this case, but, this is an inferior method compared to the other one.
 Firstly, not all things which look like files support seeking --- if one were to read from \texttt{stdin} instead of a disk file, then the seek would fail 
 (you can't rewind \texttt{stdin} all the way back to the beginning).
 Secondly, it is possible that the file could have been edited to add more numbers after your code counted them but before it finished reading them.
 Thirdly, reading the whole file twice (for larger files) is inefficient.
 Forthly, students may gravitate to this type of approach because it \emph{seems} simpler than resizing arrays rather than because they think it is a better approach.
 
 \item Note that return values of functions are checked. 
 Did the \texttt{fseek} call work? If not, there is no point continuing.
 Did the \texttt{fscanf} actually manage to read anything.
 We don't explicitly check for end of file, but that is covered by \texttt{fscanf} failures.
 Possible failures in \texttt{fprintf}, \texttt{malloc}, \texttt{free} are not tested because 
 (dealing with these failures is system specific / tricky).
\end{itemize}


\lstinputlisting[label=lst:reversenums, escapechar=@, caption=Reversing numbers]{file.c}