
\chapter{Simple output and Command Line arguments}


Compile and run the following:
\begin{codeblock}
#include <stdio.h>

int main(int argc, char** argv) {
    printf("Going to print some things %s%s%s\n");
    return 0;
}
\end{codeblock}

Although you \emph{can} just print strings directly with printf, it isn't necessarily a good idea (as shown above)\footnote{\texttt{puts()} is probably a better option}.
To understand why, we need to look at how \texttt{printf} deals with variables.
Replace Line~$4$ with
\begin{codeinline}
printf("%s\n", "Going to print some things %s%s%s ");
\end{codeinline}

The first parameter of \texttt{printf} describes what is to be printed: any sequence starting with \% is called a placeholder and will be replaced with something else when the output is printed.
Characters after the \% indicate what type of thing is to be printed (and possibly how it is to be formatted).
So \%s in the first argument tells \texttt{printf} that it should look at the next argument along, interpret it as a string and print it.
If there aren't enough arguments, or the argument is the wrong type, unexpected things may occur.

To print an integer instead of a string, use the \%d placeholder.
For example: \lstinline!printf("%d + %d = %d\n", 5, 7, 12);!
You can use any mixture of placeholders and normal characters in the \emph{format string} that you like.
To print a \% symbol, use \%\%.


Compile the following:
\begin{codeblock}
#include <stdio.h>

int main(int argc, char** argv) {
    printf("There were %d arguments given to the program %s\n",
      argc, argv[0]);
    return 0;
}
\end{codeblock}
Lines~4 and 5 show that you can break lines to improve layout and readability.
Try running the program with a variety of command line arguments.
You will see that \texttt{argv[0]} contains the name of the program (\texttt{argv} stands for ``argument values") and that \texttt{argc} gives the ``argument count'' --- how many there were.
Since \texttt{argv} always has at least one thing in it, \texttt{argc} $\geq 1$ and accessing \texttt{argv[0]} should always be safe.
But change the program to print \texttt{argv[200]} and see what happens.


By using a conditional statement, we can safely print more if it is available.
\begin{codeblock}
#include <stdio.h>

int main(int argc, char** argv) {
    printf("0: %s\n", argv[0]);
    if (argc > 1) {
      printf("1: %s\n", argv[1]);
    }
    if (argc > 2) {
      printf("2: %s\n", argv[2]);
    }
    return 0;
}
\end{codeblock}
Some type of loop would be a better way to deal with this, so let's do that now.

\section{Loops}
A loop consists of the following parts (some of which may be empty):
\begin{itemize}
 \item [Setup] --- happens once (before the loop starts).
 \item [Test]  --- should the loop repeat or stop now?
 \item [Body]  --- what the loop is supposed to do each time.
 \item [Update] --- happens after the body runs each time.
\end{itemize}

So to display command line arguments:
\begin{itemize}
 \item [Setup] --- set a counter variable (\texttt{i}) to 0.
 \item [Test]  --- is \texttt{i} less than \texttt{argc}
 \item [Body]  --- print the argument
 \item [Update] --- increase \texttt{i}
\end{itemize}

C has three types of loop structure:

The \emph{for} loop:
\begin{codeinline}
for (Setup ; Test ; Update) {
    Body
}
\end{codeinline}

The \emph{while} loop:
\begin{codeinline}
Setup
while (Test) {
    Body
    Update
}
\end{codeinline}

and the \emph{do-while} loop:
\begin{codeinline}
Setup
do {
    Body
    Update
} while(Test);
\end{codeinline}

The \emph{do-while} loop differs from the \emph{while} loop by always running the Body and Update once before the Test is performed and the loop repeats.

So we could do this:
\begin{codeblock}
#include <stdio.h>

int main(int argc, char** argv) {
    for (int i = 0; i < argc; i = i + 1) {
        printf("%d: %s\n", i, argv[i]);
    }
    return 0;
}
\end{codeblock}

\begin{exercise}
Rewrite the above to use \texttt{while} and then to use \texttt{do-while}. 
\end{exercise}

