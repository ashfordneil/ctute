
\chapter{Strings}

\begin{knowledge}
This part assumes that you have a good understanding of the materials on arrays and files. 
\end{knowledge}

So far, we have output strings (using the \%s placeholder) but apart from \texttt{argv} in main have avoided string variables and string input.
Strings are stored in arrays of char, however in C you can ask how long a string is but not how big an array is.
This is because strings use a special character with the numeric value $0$ (written as \verb!`\0'!) to mark the end of the string.
So \lstinline!strlen("") == 0! that is, the length of the empty string is zero.
This also means that \lstinline!("")[0] == '\0'!.
The \emph{null terminator} is not counted as part of the length of the string.


For example:

\begin{codeblock}
char single[2];
single[0] = 'A';
single[1] = '\0';
\end{codeblock}

would make \texttt{single}, the same as \verb!"A"!.
Note the use of \lstinline!'! around characters and \lstinline!"! to mark strings.

On the other hand,
\begin{codeblock}
char single[2];
single[0] = 'A';
single[1] = 'B';
\end{codeblock}
would not make \texttt{single} store \verb!"AB"!, rather it is not storing a proper string at all\footnote{It is actually worse than that.
The most we can say, is that we can't tell whether single stores \texttt{"AB"} or not.
This is because we don't know what follows the array in memory.
If there just happened to be a zero byte next then, when the program looked in \texttt{single} it would see a proper string.
}.
Important note: while strings are stored in arrays of char, that does not mean all arrays of char store strings.

Because strings are stored in arrays, it does not make sense to compare them with \texttt{==}.
Instead, we use standard string functions from \lstinline{<string.h>}.
Compile and run the following program:
\begin{codeblock}
#include <string.h>
#include <stdio.h>

int main(int argc, char** argv) {
    char msg[10];	
    msg[2] = '\0';
    msg[0] = 'h';
    msg[1] = 'w';
    printf("strlen(%s)==%d\n", msg, strlen(msg));
    printf("msg==\"hw\" -> %d, strcmp(msg,\"hw\")
        ==%d\n", (msg == "hw"), strcmp(msg, "hw"));
    return 0;
}
\end{codeblock}

In this case, both comparing with \texttt{==} and comparing with the string compare function (\texttt{strcmp}) both give zero/\lstinline!false!.
However, we need to be clear about what \texttt{strcmp} returns.
\begin{codeblock}
strcmp("B", "A") == -1 // arguments are in reverse order
strcmp("B", "B") == 0  // arguments are the same
strcmp("A", "B") == 1  // arguments are in correct order
\end{codeblock}
That is, strcmp returns \textbf{0} if strings are the same.

To copy strings, use \texttt{strcpy}:
\lstinline{strcpy(buffer, "hello ");}
String functions which modify a string, put the destination first.
To add to an existing string, use \texttt{strcat} (string concatenate):
\lstinline{strcat(buffer, "world");}
After executing those two statements, \texttt{buffer} will contain \lstinline!"hello world"! \emph{provided that \texttt{buffer} was big enough}.
If buffer was declared as \lstinline!char buffer[12]! or bigger then we are ok.
If it was smaller (eg 2 chars), then the string functions would write off the end of the array and corrupt other variables or crash (possibly)\footnote{This possibly is the worst part.
The program \emph{might} crash or it might look fine.
The behaviour might change depending on which system you ran the program on.
Undefined and unpredictable behaviour like this can be very hard to find and debug.}.

What if you want to put numbers into a string? 
The simplest way is with \emph{s}\texttt{printf}:
\begin{codeblock}
char buffer[20];
sprintf(buffer, "%d + %d == %d", 3, 5, 3 + 5);
\end{codeblock}
Will result in \texttt{buffer} storing \lstinline!"3 + 5 == 8"!.
As with all standard functions which modify strings, \texttt{sprintf} will put a \lstinline!'\0'! on the end of the string it makes.
There is also an \texttt{sscanf} to read values out of a string instead of from a \texttt{FILE*}.

If you are making a string which will be passed outside the current function, it is better to use dynamic arrays:
\begin{codeblock}
char* getMessage1(void) {
    char* msg = malloc(sizeof(char) * 11);
    sprintf(msg, "%d + %d == %d", 3, 5, 3 + 5);
    return msg;
}

char* getMessage2(void) {
    char msg[11];
    sprintf(msg, "%d + %d == %d", 3, 5, 3 + 5);
    return msg;		// returning pointer to local variable
}
\end{codeblock}

\texttt{getMessage2} is dangerous because the memory used for array \texttt{msg} be used for other things once the function returns.
Since \texttt{getMessage1} allocates it's memory is on the heap and so won't be reused until it is \texttt{free}d.

\section*{String summary}
\begin{itemize}
 \item Strings are stored in \texttt{char} arrays and must be terminated properly.
 \item Standard functions will take care of \lstinline!'\0'! for you. In your own code, you need to look after it.
 \item C won't check that the underlying storage is big enough.
 \item \texttt{strlen(s)} --- length of \texttt{s}.
 \item \texttt{strcpy(d, s)} --- make \texttt{d} a copy of \texttt{s}
 \item \texttt{strcat(d, s)} --- copy \texttt{s} onto the end of \texttt{d}.
\end{itemize}


\begin{exercise}

\begin{enumerate}
\item
Write a function called \texttt{mystrlen} which does the same job as the standard \texttt{strlen} function.
You are not permitted to call any standard C string functions.

Test it with the following:
\begin{lstlisting}[numbers=none]
#include <stdio.h>

// your code here

int main(int argc, char** argv) {
    if ((mystrlen("hello") == 5) && (mystrlen("") == 0)) {
        printf("Ok\n");
    } else {
        printf("No\n");
    }
    return 0;
}
\end{lstlisting}

\item Write your own version of \texttt{strcpy} (mystrcpy) which does the same job.
You are not permitted to call any standard C string functions.
Write a test program to demonstrate your function is correct.

\item Write a function \lstinline!char* stringadd(const char* s1, const char* s2)! which takes two strings and 
returns a new string which contains \texttt{s1} followed by \texttt{s2}.
You are not permitted to call any standard C string functions.
Write a test program to confirm that your function is correct.

\item Write a function which reads a line of any length from a file, and returns it as a string.
You are not permitted to use \texttt{fgets}.
Write a test program to confirm that your function is correct.
 
\end{enumerate}
 
\end{exercise}
