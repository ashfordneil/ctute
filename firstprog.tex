
\chapter{First program}

\begin{knowledge}
This part assumes that you are familiar with strings and arrays from some other programming language.
\end{knowledge}

Compile and run the following:
\begin{codeblock}
#include <stdio.h>

int main(int argc, char** argv) {
    printf("It begins\n");
    return 0;
}
\end{codeblock}

\begin{itemize}
 \item[1] \lstinline!#include! is a ``preprocessor directive''\footnote{as opposed to a C statement or declaration} which tells \texttt{gcc} to 
 consult another file before going any further.  If you remove this line, you will see that \texttt{printf} becomes a problem.
 This is because the \texttt{stdio.h} file is what tells \texttt{gcc} about the \texttt{printf} function.
 
 The \texttt{<>} around \texttt{stdio.h} indicate that it is a system include rather than a header we created ourselves\footnote{
 More accurately, \texttt{<>} indicate which paths the compiler should look for the file in.}.
 For our own headers, put the filename in double quotes.
 
 \item[3] begins a function definition\footnote{A definition tells the compiler both that the function exists(called declaration), and what the function does.}.
 All standard C programs must have a function called \texttt{main}.
 Unless your system has special requirements, you should use the signature\footnote{A function's signature gives the function's name, its return type and the type and order of parameters.} shown here.
 We have not discussed types yet (we'll look at types in Section~\ref{sec:types}) but for now, \texttt{int} is an 
 integer\footnote{A whole number, either positive or negative} while \texttt{char} on its own represents a character.
 The first \texttt{int} means that our \texttt{main} function returns an int.
 The parts in parentheses are the parameters for the function.
 The first one is an integer and the second one has something to do with characters.
 Without explaining why now, the second parameter is an array of strings.
 Note that this program doesn't actually make use of the parameters it declares.
 
 \item[4] This is a function call. It has the name of a function \texttt{printf} followed by arguments in parentheses.
 Text in double quotes represents a string. 
 So this line passes a string to the \texttt{printf} function.
 
 \item[5] indicates which value the function returns/sends back to its caller.
 For \texttt{main}, the return value is sent back to the operating system as the program's exit status.
\end{itemize}

