
\chapter{Functions and Passing types}

A function declaration in C consists of a return type, a function name and a parameter list.
For example:
\lstinline!int main(int argc, char** argv)!
A function definition, has that information followed by a function body. 
If no body is given, the declaration should be followed by a semi-colon.

So \lstinline!void show_greeting(void);!  is just a declaration.
It tells the compiler that the \texttt{show\_greeting} function exists but not how to do it.
\begin{lstinline}
void show_greeting(void) {
    printf("Hello\\n");
}
\end{lstinline}
Is a definition (which also counts as a declaration).

Unlike many other languages, to tell the compiler that a function has no parameters, you 
need to write \lstinline!(void)! for the parameter list instead of just \lstinline!()!.
A return type of \lstinline!void! means that the function does not return anything.

If a function has parameters, then the type of each parameter needs to be given explictly.
As opposed to declaring normal variables, where you can do this: \lstinline!int i,j,k!.

For example, the following function returns the maximum of two integers.
\begin{lstlisting}
int maximum(int a, int b) {
    if (a < b) {
        return b;
    } else {
        return a;
    }
}
\end{lstlisting}

Exercise: Write a function which \emph{returns} the maximum of three integers and a program to test it.

Enter and compile the following:
\begin{lstlisting}
#include <stdio.h>

void swap(int p, int q) {
    int temp;
    printf("p=%d q=%d\n", p, q);
    temp=p;
    p=q;
    q=temp;
    printf("p=%d q=%d\n", p, q);
}

int main(int argc, char** argv) {
    int a=3, b=4;	// variables can be initialised as they are declared.
    swap(a,b);
    printf("a=%d b=%d\n", a, b);
    return 0;
}
\end{lstlisting}

Note that the paramters listed for the swap function are called \texttt{p} and \texttt{q} but the arguments passed to
the function when it is called are from variables \texttt{a} and \texttt{b}.
This does not matter. 
There is no relationship between parameter names (used inside a function) and the names used to pass arguments into 
the function.
If we had named the parameters \texttt{a} and \texttt{b}, that would not make a link between the variables inside \texttt{swap} 
and inside \texttt{main}.

Bear in mind that even though the compiler doesn't pay any notice if the variables are the same, people might.
Try to avoid switching variable names around for no reason.

When you have run the above program, you will notice that it doesn't seem to work.
The variables in \texttt{main} still have their original values.
However, \texttt{p} and \texttt{q} seemed to have swapped over just fine.

The reason for this is that arguments to C functions are passed \emph{by value}.
That is, the \emph{value} of each argument is copied into the paramter variables inside the function.
So, the function swaps values in variables but that doesn't change anything outside of the function.

To demonstrate this further, try changing the swap call to \lstinline!swap(2, (5*7)-6)!.

To allow functions to modify their arguments, the you need use \emph{pass by reference} which C doesn't really have.
Instead we make use of \emph{pointers}.
A pointer is not a way to refer to memory by its address rather than by variable name.
You can get the address of any variable using the \lstinline!&! operator.
So \lstinline!&length! gives a pointer to the \texttt{length} variable.

Now for the fun part.
Pointers are values as well and can be stored in variables.
\lstinline!int pl=&length! won't work though because \lstinline!&length! isn't an \texttt{int}\footnote{Yes, technically underneath it is an unsigned integer value of some type (probably \texttt{unsigned long}).
But it is more helpful for programmers when the compiler can notice that this wasn't supposed to be a normal number.}, 
it is a \emph{pointer-to-\texttt{int}} (or an \emph{\texttt{int}-pointer}, the names are equivalent).

In C, pointers are declared by writing \texttt{*} between the type-being-pointed-to and the name.
\begin{lstlisting}
int* lp; // ok but a meaningful name would be better
int* lpointer; // longer but not necessarily any more meaningful
int* i, j;   // Warning: this does not mean what you think
int *k;	// the * can also go against the variable name
int*p;	// also legal. C doesn't care about spaces here
int * q; // Now you're just being silly. (Still legal though).
\end{lstlisting}
The last two lines are examples of legal code which is unnecessarily hard to read.
Remember that people (including you) will need to read your code later and you don't want that to be difficult.

Line~3 illustrates a tricky part of pointer declarations.
Regardless of where you put the \texttt{*}, only the next variable name will be a pointer.
So, Line~3 declares \texttt{i} to be a pointer to an \texttt{int} and \texttt{j} to be an ordinary (non-pointer) \texttt{int}.

Important note:
there is no compelling argument as to which \texttt{*} position is more correct (\texttt{int* x} or \texttt{int *x}).
Lot's of people will try to tell you that there is though.
It is more important that you use a form which you are comfortable with and (perhaps more importantly) is consistant with 
the rest of the code you are writing.
Most importantly, you need to conform with whatever style rules are in force for your organisation.

Modify the swap declaration to be:
\lstinline!void swap(int* p, int* q)!
and try to compile.

The problem is that the function now expects pointers and we are passing in \texttt{int}s.
So change the swap \emph{call} to \lstinline!swap(&a,&b);!.
This means that the function expects pointers and we are giving it pointers.
The code inside the function is not right yet though.
Line~3 tries to print out the paramters but we don't care what the addresses of the variables are, we want to know what is
at the end of the pointers.
Remember that these pointers are pointing at \texttt{a} and \texttt{b} and we want to swap what is in \texttt{a} and \texttt{b}.
So we need the \emph{dereference} (follow-the-pointer) operator.
This is a \texttt{*} before the pointer.
Line~3 would become:
\lstinline!printf("p=%d q=%d\n", *p, *q);!

In fact, all the other references to \texttt{p} and \texttt{q} need to be changed to \texttt{*p} and \texttt{*q}.
Compile and run to see if it worked.

So what happened here?
Suppose \texttt{a} and \texttt{b} are stored in locations \lstinline{xFFFF0000} and \lstinline{xFFFF0008} respectively.
\begin{itemize}
 \item \lstinline{swap(&a, &b)} will become \lstinline{swap(xFFFF0000, xFFFF0008);}.
 \item Those pointer \emph{values} will be \emph{copied} into the function variables \texttt{p} and \texttt{q}.
 \item \lstinline!temp=*p! would become \lstinline!temp=*(xFFFF0000)! ie \texttt{temp=} the value stored at location 
 \lstinline!xFFFF0000!.  That is, \lstinline!temp=3;!
\end{itemize}

So, the pointers were passed by value (copied) but a copy of a pointer is just as good at accessing the original thing.

Main points to remember:
\begin{enumerate}
 \item If you want to change something outside a function from inside the function, you may need to use a pointer.
 \item Pointers have two ends. eg \lstinline!int* fred!.   \texttt{fred} is the pointer and \texttt{*fred} is the thing 
 being pointed at.
 \item Pointers to different types of things are different. \texttt{int* c} and \texttt{char* i} are not the same type of thing.
 \item There is a special pointer written as \texttt{NUL} or \texttt{0} which points to nothing.
 Trying to follow a \emph{null pointer} leads to sadness.
 \item Just like other variables in C, pointer variables are not guaranteed to be initialised.
 That is, they do not point anywhere sensible unless you tell them to.
\end{enumerate}


Exercise: suppose you have 3 variables \texttt{a, b, c}.
Write a function which you can use to switch the values around so that 
\texttt{a} takes the value of \texttt{b}, \texttt{b} takes \texttt{c}
and \texttt{c} takes the original value of \texttt{a}.

So, if \lstinline{a=0, b=1, c=2} before the call, then after the call:
\lstinline{a=1, b=2, c=0}.
Write a program to test your function.



