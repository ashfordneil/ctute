
\chapter*{Glossary}

\begin{description}
 \item[\hypertarget{defn:argument}{Argument}] a value supplied to a function to tell it what it should be doing.
 For example (if \texttt{sqrt} computes square roots, then in \lstinline{sqrt(36)}, $36$ is an argument but the same function
 could be called later with a different argument.
 
 Note that arguments are distinct from the names given to the variables the arguments are stored in inside the function (see \hyperlink{defn:parameter}{Parameters}).
 \item[\hypertarget{defn:array}{Array}] A variable, which contains multiple values intead of just one. 
 Individual values are accessed by their \texttt{index} within the array.
 In most C derived languages this is indicated with brackets.
 eg:
    \lstinline!scores[17]!
    
 In C, all the values have the same type and the size of the array is fixed 
 when it is created.
 
 \item[\hypertarget{defn:parameter}{Parameter}] A piece of information which a function needs in order to do its job.
 For example: the function for computing a square root could be declared as:
 \lstinline!double sqrt(double x)!.
 In order to work, the function needs to know which number it should be taking the square root of.
 In this case, that information will be stored in a variable called \texttt{x}.
 Note that parameters are distinct from \hyperlink{defn:argument}{Arguments} which are the values given in a specific call.
 
 \item[\hypertarget{defn:string}{String}] A sequence of characters which can be manipulated as a unit.
 In C derived languages, they are usually written in code surrounded by double quotes\footnote{The widely known exception 
 is Python, where single or double quotes can be used for strings.}.
 eg:
 \lstinline!"Welcome to C"!
\end{description}
