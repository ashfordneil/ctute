
\chapter{Structs}


% Should go after pointers so we can explain about passing them by value

Often when you are programming you will have a number of variables all related to the same task.
In a language like python or java, you may use a class to help signify this grouping, which can also have methods which operate on instances of the class (objects).
In C there is no concept of a class, however it does provide a way to group variables into a common place.
This is called a struct.

\section{Defining a Struct}

The syntax for defining a struct is shown below.
This defines a struct whose name is myStruct.
The struct contains 3 members, an \texttt{int} called \texttt{a}, a \texttt{char} called \texttt{b} and an \texttt{int *} called \texttt{c}.

\begin{codeblock}
struct myStruct {
    int a;
    char b;
    int *c;
};
\end{codeblock}

\section{Declaring and accessing members}

Structs can be used in place of any other type.
This means you can have a struct variable, and also a struct pointer, shown below.

\begin{codeblock}
/* var1 is of type struct myStruct */
struct myStruct var1; 
/* var2 is of type struct myStruct pointer */
struct myStruct *var2; 
\end{codeblock}

If you have a variable of type struct something, then you can access the members using the \texttt{.} operator.
For example using \texttt{var1} from above we could have

\begin{codeblock}
int x = var1.a;
char y = var1.b;
int *z = var1.c;
\end{codeblock}

If you have a pointer to a struct, then you can access the members using the \texttt{->} operator.
For example using \texttt{var2} from above

\begin{codeblock}
int x = var2->a;
char y = var2->b;
int *z = var2->c;
\end{codeblock}

You can also assign to struct members using the same syntax.

\section{Passing structs to functions}

Often you might be in a situation where you have a struct, and you want to call a function which will use and update some part of that struct.
A mistake people often make here is that they assume structs are some sort of special type like in java/python where they are passed by reference by default.
This is not the case.
If you pass a struct to a function, that function will be working on a \emph{copy} of the struct.
Whatever you do to it in the function will not be reflected in the calling function's version of the struct.
You can return structs from a function to allow the values to be updated, but that involves a lot of copying you probably don't need.

As a general rule, if you're passing a struct to a function it's likely you want to be passing a pointer to that stuct.
As was previously mentioned in the pointers section, a pointer lets the function know where \emph{your} copy of the stuct it and allows it to update it.
The follow example demonstrates how this works.

\begin{codeblock}
void
structFunc(struct myStruct *s)
{
    /* Note the use of -> because we have a pointer */
    s->a = 10;
}

void
otherFunc(void)
{
    struct myStruct s;

    /* Here we use . because it's not a pointer! */
    s.a = 5;
    printf("s.a before: %d\n", s.a); // prints 5
    structFunc(&s); // make a pointer to s
    printf("s.a after: %d\n", s.a); // prints 10
}
\end{codeblock}

\section{Self referencing structs}

An interesting thing to consider is whether a struct can contain itself.
If you think about this, a struct containing itself creates an infinite recursion when computing the size, so it is not allowed.
However a struct can contain a \emph{pointer} to itself.
Why?
Because the size of a pointer is always known by the compiler.
The size of the thing it points to has no impact on the memory needed to store an address.

Consider the example below.
We have a struct type called \texttt{node} which has 2 ints and then a pointer to another struct of the same type.
You will see how this can be used to implement useful data structures in a later section.

\begin{codeblock}
struct node {
    int a;
    int b;
    struct node *next;
};
\end{codeblock}

\section{The size of a struct}

In the above section we mentioned the compiler computing the size of a struct.
If you want to use the malloc rule that was introduced early, you need to be able to find the size of the struct in bytes.
Fortunately this works just like every other type!

\begin{codeinline}
struct myStruct *s = malloc(sizeof(stuct myStruct) * 1);
\end{codeinline}

One thing you need to be careful of when working with structs is making assumptions about how big it will end up being.
Consider the example below.
It may be intuitive to think the total size of this struct will be 5 bytes, but that is unlikely to be true unless you ask the compiler really nicely.
Modern CPU's in general require data to be aligned on certain memory address boundaries in order to load them most efficiently, and the compiler will respect this.
You should always ask the compiler using \texttt{sizeof} because it is the one who ultimately decides the size that is best for the CPU your code is running on.
You may find that the size of the struct below is actually 8 bytes.

\begin{codeblock}
struct s {
    char a; // 1 byte
    int b; // 4 bytes
};
\end{codeblock}

