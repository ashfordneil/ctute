
\chapter{How to attack a spec}

% Here is where we discuss how to think about programming tasks.
% How do you decide what functions you need?
% How do you break down a task

The following is a specification for a programming task.
You are not expected to actually code this.
The goal instead is to look at how it could be broken into managable pieces 
for a sane development strategy.

Read through the following assignment spec.
In Section~\ref{sec:breakdown}, we'll discuss it.

\section{Introduction}
Your task is to write an \emph{ANSI-C} program (called match) which
allows the user to play a game.

match will display a grid of cells, each containing a symbol.
The user will nominate a cell (by entering its coordinates).
If the symbol in the cell matches the symbol in
any of the cells directly above, below, left or right, then the selected cell and any of its matching neighbours
will be removed.
This effect is also applied to neighbours of matching neighbours (and so on).
As an example, suppose the grid contents (surrounded by a border) were:

\begin{center}
\begin{verbatim}
+-----+
|ZAAAA|
|ZBBCC|
|ZWaaX|
|XXXXX|
|Zcdef|
+-----+
\end{verbatim}
\end{center}
and the cell \texttt{3 1} was chosen (that is, Row~\texttt{3} and Column~\texttt{1}, with the top left being \texttt{0 0}).
In this case, all of Row~\texttt{3} would be removed as well as the last value on Row~\texttt{2}.

\pagebreak
\begin{center}
\begin{verbatim}
+-----+
|ZAAAA|
|ZBBCC|
|ZWaa.|
|.....|
|Zcdef|
+-----+
\end{verbatim}
\end{center}

If a move leaves a gap in the middle of a column, the symbols above it will be moved down to fill the
the gap.

\begin{center}
\begin{verbatim}
+-----+
|.....|
|ZAAA.|
|ZBBCA|
|ZWaaC|
|Zcdef|
+-----+
\end{verbatim}
\end{center}

If \texttt{4 1} was chosen, nothing would happen because the `c' in that cell is not touching any other `c's.
If a removal leaves one or more whole columns empty, any columns to the right of the empty ones, will be
moved over to fill the gap.
For example, if \texttt{1 0} was chosen in the above, the result would be:

\begin{center}
\begin{verbatim}
+-----+
|.....|
|AAA..|
|BBCA.|
|WaaC.|
|cdef.|
+-----+
\end{verbatim}
\end{center}

If the grid looked like:
\begin{center}
\begin{verbatim}
+-----+
|AAAAA|
|ZABAC|
|ZAaAX|
|XAXAX|
|ZAdAf|
+-----+
\end{verbatim}
\end{center}
and \texttt{0 0} was chosen, the result would be:
\begin{center}
\begin{verbatim}
+-----+
|.....|
|ZBC..|
|ZaX..|
|XXX..|
|Zdf..|
+-----+
\end{verbatim}
\end{center}

The game ends when either all the symbols are removed, or there are no more legal moves.


\subsection{Invocation}\label{sec:invoc}
\noindent
When run with no arguments, match should print usage instructions to stderr:\\
\begin{verbatim}
Usage: match height width filename
\end{verbatim}
and exit (see the error table).

\texttt{height} and \texttt{width} must be positive integers greater than 1 and less than \texttt{1,000}.
\texttt{filename} must be a path to a readable file containing a valid grid (see later) with dimensions matching \texttt{height}
and \texttt{width}.
See the error table for what to do if these constraints are not met.

When run with valid arguments, match should read in the map file.
It should then display the grid.
If there are no legal moves, it should exit with the appropriate message; otherwise it should display the prompt:
\begin{verbatim}
>
\end{verbatim}
(note, a space follows the prompt).
The user should then enter either:
\begin{itemize}
 \item the row and column they wish to remove (space separated). eg \texttt{3 2}
 \item \texttt{w} followed immediately by a filename(or path). eg \texttt{whello}
 This will result in the current grid (including the border) being written to a file called ``hello''.
\end{itemize}
If there is no more input from the user, a message should be displayed (see error table) and
the program should exit normally.
If the user enters anything else, the prompt should be displayed again.

After the user has entered a legal (non-w) move, the grid should be updated and displayed again.
If the game is over, display the appropriate message and exit.
If not, display the prompt again and wait for another move.

\subsection{Input file format}
A valid grid file will consist of rows of symbols (and blanks) terminated by newlines (\verb|\n|).
You may use whichever characters you want for grid symbols except the following:
\begin{itemize}
 \item \verb|\0|          --- the null character. This should trigger an error in your program.
 \item \verb|\n|   --- newlines will always be interpretted as the end of a line.
 \item \verb|.| --- a dot indicates a blank. You can include these in your input but the gaps will not be filled until
 a valid move has been completed.
\end{itemize}

\subsection{Errors and messages}\label{sec:errs}
When one of the conditions in the following table happens, the program should print the error message and exit with
the specified status.
All error messages in this program should be sent to \emph{standard error}.
Error conditions should be tested in the order given in the table.
All messages are followed by a newline.

\begin{center}
\begin{tabular}{|p{4cm}|c|p{5.5cm}|}\hline
\textbf{Condition} & \textbf{Exit Status} & \textbf{Message} \\\hline
Program started with incorrect number of arguments & 1 & Usage: match height width filename \\\hline
Invalid grid dimensions & 2 & Invalid grid dimensions \\\hline
Cannot open grid file& 3 & Invalid grid file \\\hline
Error reading grid. Eg: bad chars in input, not enough lines, short lines & 4 & Error reading grid contents \\\hline
\end{tabular}
\end{center}
For a normal exit, the status will be \texttt{0} and no special message is printed.

There are a number of conditions which should cause messages to be displayed but which should not immediately terminate the program.
These messages should also go to standard error.

\begin{center}
\begin{tabular}{|p{5.0cm}|c|p{4cm}|}\hline
\textbf{Condition} & \textbf{Action} & \textbf{Message} \\\hline
Error opening file for saving grid & Prompt again & Can not open file for write \\\hline
Save of grid successful & Prompt again & Save complete \\\hline
End of input while waiting for user input & exit program normally & End of user input \\\hline
\end{tabular}
\end{center}

If after a move, all the symbols have been removed, the program should display ``Complete'' to stdout and exit normally.
If the game has reached a configuration where completion is impossible and there are no more legal moves,
the program will display ``No moves left'' to stdout and exit normally.

\subsection{Compilation}
Your code must compile with command:\\
\texttt{gcc -Wall -ansi -pedantic ass1.c -o match}

\section{Marks}

\subsection{Functionality criteria}

\begin{center}
\begin{itemize}
\item Command args --- correct response to
\begin{itemize}
 \item incorrect number of args
 \item invalid dimensions
 \item invalid grid filename
\end{itemize}
\item Correctly display initial grid and prompt
\item Reject illegal moves on the initial grid
\item Correctly process a single move:
\begin{itemize}
 \item Single removal from top line
 \item Single removal of whole rightmost column
 \item Single removal covering multiple rows/columns
\end{itemize}
\item Detect no more legal moves
\item Correctly save grid
\item Play complete games with multiple removals
\end{itemize}
\end{center}

\hrule
\section{Comments}\label{sec:breakdown}

Reading the spec we can see some points:
\begin{enumerate}
 \item There is a grid which can store letters.
 \item We don't know how big the grid needs to be until the program starts running (so no fixed sized variables).
 \begin{itemize}
  \item The grid does not change size once the game starts though.
 \end{itemize}
 \item There are rules about removing letters from the grid
 \item There is something about files and file formats
\item There are instructions about command line arguments.
 \item There are instructions about error conditions and what to do when they occur.
 \item There is something about how it should compile. \emph{Do not ignore instructions about the intended target platform. 
 If it does not work on the client's machine, it does not work\footnote{unless the client has a broken environment, but that
 is a separate issue.}.}
 \item Finally there are some sort of criteria.
\end{enumerate}

So where to start?
It seems like taking things out of the grid seems to be important so should we start there?
Probably not.
When working on code, you should be thinking about how you are going to make sure that what you have written is correct.
That is, ``how are you going to test that it complies with the spec?''
Do not fall for the idea that you can just test everything at the end and it will be less work\footnote{This is almost always wrong.}.
The more code you need to search, the harder finding bugs will be.

A better approach is to try to look for smaller chunks of functionality.
That you can write \emph{and test} without needing lots of other code you don't have yet.
When you have completed (and tested) a new chunk of functionality, record that version somewhere (eg in version control)
and then move on.

For example, look at the command line argument checking.
In Section~\ref{sec:invoc} it says that you need to display a message when the program is run with no arguments.
This is easy to write and check:
\begin{itemize}
 \item Run with no arguments and check message.
 \item Run with arguments and no message.
\end{itemize}

Section~\ref{sec:errs} describes some extra errors you need to check to do with command line arguments.
Some have to do with processing the contents of a file, so that might need to wait, but the rest of them
should be reasonable.
Again, test each possibility and save that version.
The reason for storing your code after each bit works is that if you break it later\footnote{This raises
another point. Just because something worked before doesn't necessarily mean it still does.
You will want to recheck earlier tests from time to time as well. This is where
automated testing (sometimes mistakenly called unit testing) can be useful.} (adding more features, you have a 
working version you can go back to).

Now you should have some code you can rely on (and you can easily specify different sizes when testing other parts later).

What to do next may be a bit less obvious.
You could try the file reading code (checking for errors) but you don't have anywhere to put any grid information once you've
read it.
In fact, the next thing to do would probably be the representation of the grid but how to test it?
So you would need to work out how to to store the grid \emph{and} how to display it.
Start with creating and displaying a blank grid.
Then try modifying the grid in code and displaying again to see if things appear in the correct place.
This last point, can only be a temporary modification since a working program would not do this, so remove it once you 
are sure it works\footnote{Another strategy you can employ is to have one or more files which have your functions in them and then
use another file which has test functions in it and call the normal functions from there.}.

Following from there you have a number of options.
Removal of letters? Saving? Loading? Game over detection?
The first two are possible, but you would need to add test code to create various board states to test with, so being able 
to have your testing states in files and pulling them in with load would seem to be easier.

When it comes to dealing with removing letters, rather than trying to solve the whole thing in one go.
Try to identify simple subcases, eg an entire column is removed and build up from there.

